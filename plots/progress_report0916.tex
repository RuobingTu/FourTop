\documentclass{beamer}

% This file is a solution template for:

% - Talk at a conference/colloquium.
% - Talk length is about 20min.
% - Style is ornate.



% Copyright 2004 by Till Tantau <tantau@users.sourceforge.net>.
%
% In principle, this file can be redistributed and/or modified under
% the terms of the GNU Public License, version 2.
%
% However, this file is supposed to be a template to be modified
% for your own needs. For this reason, if you use this file as a
% template and not specifically distribute it as part of a another
% package/program, I grant the extra permission to freely copy and
% modify this file as you see fit and even to delete this copyright
% notice. 


\mode<presentation> %
{
  %\usetheme{Warsaw}
  % or ...
%\usetheme{Madrid}%https://www.overleaf.com/learn/latex/Beamer#Reference_guide
  \usetheme{AnnArbor}

  \setbeamercovered{transparent}
  % or whatever (possibly just delete it)
}

\usepackage{array}
\usepackage{tabularx} 
\usepackage{underscore}
%\usepackage{multirow}
%\usepackage{longtable}
%\usepackage{subfigure}not found
%\usepackage{lipsum}
%\usepackage[demo]{graphicx}
\usepackage{graphicx}
\usepackage{graphicx,subcaption}
\usepackage{caption}
\usepackage{subcaption}
\graphicspath{ {/publicfs/cms/user/huahuil/TauOfTTTT/2016v1/v2_NewNtupleAfterEventSelection/Plots/add_all_bg/} }

\usepackage[english]{babel}
% or whatever

\usepackage[latin1]{inputenc}
% or whatever

\usepackage{times}
\usepackage[T1]{fontenc}
% Or whatever. Note that the encoding and the font should match. If T1
% does not look nice, try deleting the line with the fontenc.


\title[IHEP Group Meeting] % (optional, use only with long paper titles)
{Progress Report on Tau Final States of TTTT}
%\subtitle
%{Include Only If Paper Has a Subtitle}
\author[Huiling Hua] % (optional, for multiple authors)
%{A.~B.~Arthur\inst{1} \and J.~Doe\inst{2}}
{Huiling Hua\inst{1} \and Hongbo Liao\inst{1} \and Hideki Okawa\inst{2} \and Yu    Zhang\inst{2}}
%\author{Huiling Hua}
%\institute{IHEP}
\institute[IHEP] % (optional)
{
  \inst{1}%
 % Faculty of Physics\\
 % Very Famous University
    IHEP
  \and
  \inst{2}%
    Fudan University
}
% - Give the names in the same order as the appear in the paper.
% - Use the \inst{?} command only if the authors have different
% - Use the \inst command only if there are several affiliations.
% - Keep it simple, no one is interested in your street address.
\date[IHEP 2020] % (optional, should be abbreviation of conference name)
{IHEP Group Meeting, 2020}
% - Either use conference name or its abbreviation.
% - Not really informative to the audience, more for people (including
%   yourself) who are reading the slides online
\subject{Physics Analysis}
% This is only inserted into the PDF information catalog. Can be left
% out. 

% If you have a file called "university-logo-filename.xxx", where xxx
% is a graphic format that can be processed by latex or pdflatex,
% resp., then you can add a logo as follows:
% \pgfdeclareimage[height=0.4.1cm]{university-logo}{university-logo-filename}
% \logo{\pgfuseimage{university-logo}}

% Delete this, if you do not want the table of contents to pop up at
% the beginning of each subsection:
\AtBeginSubsection[]
{
  \begin{frame}<beamer>{Outline}
    \tableofcontents[currentsection,currentsubsection]
  \end{frame}
}

% If you wish to uncover everything in a step-wise fashion, uncomment
% the following command: 
%\beamerdefaultoverlayspecification{<+->}


\begin{document}

\begin{frame}
  \titlepage
\end{frame}

\begin{frame}{Outline}
  \tableofcontents
  % You might wish to add the option [pausesections]
\end{frame}


% Structuring a talk is a difficult task and the following structure
% may not be suitable. Here are some rules that apply for this
% solution: 

% - Exactly two or three sections (other than the summary).
% - At *most* three subsections per section.
% - Talk about 30s to 2min per frame. So there should be between about
%   15 and 30 frames, all told.

% - A conference audience is likely to know very little of what you
%   are going to talk about. So *simplify*!
% - In a 20min talk, getting the main ideas across is hard
%   enough. Leave out details, even if it means being less precise than
%   you think necessary.
% - If you omit details that are vital to the proof/implementation,
%   just say so once. Everybody will be happy with that.

\section{Motivation}
\subsection{Introduction to 4 Tops Process}

%\begin{frame}{Make Titles Informative. Use Uppercase Letters.}{Subtitles are optional.}
%\begin{frame}{4 Tops Process}%{Sunbtitles are optional.}
%  % - A title should summarize the slide in an understandable fashion
%  %   for anyone how does not follow everything on the slide itself.
%  \begin{itemize}
%  \item
%    1% Use \texttt{itemize} a lot.
%  \item
%    2%Use very short sentences or short phrases.
%  \end{itemize}
%\end{frame}


%\subsection{Previous Work}


\section{CutBased Selection}


\begin{frame}{Samples and Conditons}
    \begin{itemize}
    \item
    CMSSW version: CMSSW_10_2_20_UL
    \end{itemize}
    \begin{table}[htbp] %[h]
    \centering
%    \small
    \footnotesize%Use \footnotesize for a 20% (linear) reduction in font size
    \setlength\tabcolsep{2pt}%Reduce the amount of intercolumn whitespace
%    \begin{tabular}{|l | l | c|}
    \resizebox{\textwidth}{!}{% 
        \begin{tabular}{|l | l |>{$}c<{$}| l|} 
         \hline
         Process & Sample Name & Cross Section[pb] & notes \\% [0.5ex] %\[1ex]This adds extra space to the cell
         \hline
         \hline
         TTTT    & TTTT_TuneCUETP8M2T4_13TeV-amcatnlo-pythia8               & 9.103\times10^{-3} &  \\ %9.103e-03 +- 1.401e-05 pb
         TTJets  &  TTJets_TuneCUETP8M2T4_13TeV-amcatnloFXFX-pythia8        & 7.467\times10^{2}  &  \\   %7.467e+02 +- 2.820e+00 pb
         TTGJets &  TTGJets_TuneCUETP8M1_13TeV-amcatnloFXFX-madspin-pythia8 &3.773\times10^{0}   &  \\      %3.773e+00 +- 1.178e-02 pb
         ttZJets &  ttZJets_13TeV_madgraphMLM-pythia8                       & 6.559\times10{-1}  &  \\      %6.559e-01 +- 5.438e-04 p
         ttWJets &  ttWJets_13TeV_madgraphMLM                               & 2.014\times10^{-1} &  \\     %2.014e-01 +- 2.461e-03 pb
         ttH     &  ttH_4f_ctcvcp_TuneCP5_13TeV_madgraph_pythia8            & 3.372\times10^{-1} &  \\ %3.372e-01 +- 6.246e-05 pb
         ttbb    &  ttbb_4FS_ckm_amcatnlo_madspin_pythia8                   &1.393\times10^{1}   & major \\     %1.393e+01 +- 3.629e-02 pb
         WZ      &  WZ_TuneCUETP8M1_13TeV-pythia8                           & 2.343\times10^{1}  & major \\      %2.343e+01 +- 1.049e-02 pb
%         WW&  WW_TuneCUETP8M1_13TeV-pythia8 & 6.430\times10^{1}&  \\      %6.430e+01 +- 2.817e-02 pb
                 &  WWTo2L2Nu_DoubleScattering_13TeV-pythia8                & 1.697\times10^{-1} &  \\%1.697e-01 +- 2.618e-04 pb
         WpWpJJ  &  WpWpJJ_EWK-QCD_TuneCUETP8M1_13TeV-madgraph-pythia8      & 5.390\times10^{-2} &  \\      %5.390e-02 +- 2.905e-05 pb
         ZZ      &  ZZ_TuneCUETP8M1_13TeV-pythia8                           & 1.016\times10^{1}  &  \\     %1.016e+01 +- 5.141e-03 pb
         WG      &  WGJets_MonoPhoton_PtG-40to130_TuneCUETP8M1_13TeV-madgraph& 1.269\times10^{1} & major \\      %1.269e+01 +- 1.038e-02 pb
         ZG      &  ZGJetsToLLG_EW_LO_13TeV-sherpa                          & 1.319\times10^{-1} &  \\       %1.319e-01 +- 1.454e-04 pb
         WWW     &  WWW_4F_TuneCUETP8M1_13TeV-amcatnlo-pythia8              & 2.086\times10^{-1} &  \\        %2.086e-01 +- 2.024e-04 pb
         WWZ     &  WWZ_TuneCUETP8M1_13TeV-amcatnlo-pythia8                 & 1.651\times10^{-1} &  \\       %1.651e-01 +- 1.724e-04 pb
         WWG     &  WWG_TuneCUETP8M1_13TeV-amcatnlo-pythia8                 & 2.147\times10^{-1} &  \\      %2.147e-01 +- 2.206e-04 pb
         ZZZ     &  ZZZ_TuneCUETP8M1_13TeV-amcatnlo-pythia8                 & 1.398\times10^{-2} &  \\      %1.398e-02 +- 1.496e-05 pb
         WZZ     &  WZZ_TuneCUETP8M1_13TeV-amcatnlo-pythia8                 & 5.565\times10^{-2} &  \\      %5.565e-02 +- 5.500e-05 pb
%        WZG&  WZG_TuneCUETP8M1_13TeV-amcatnlo-pythia8 & 4.123\times10^{-2}&  \\      %4.123e-02 +- 4.251e-05 pb
%        WGG&  WGG_5f_TuneCUETP8M1_13TeV-amcatnlo-pythia8 & 1.819\times10^{0}&  \\      %1.819e+00 +- 5.227e-05 pb
%       WGG &  WGGJets_TuneCUETP8M1_13TeV_madgraphMLM_pythia8 & 1.711\times10^{0}&  \\      %1.711e+00 +- 1.753e-03 pb
%        ZGG&  ZGGJets_ZToHadOrNu_5f_LO_madgraph_pythia8 & 3.717\times10^{-1}&  \\      %3.717e-01 +- 4.788e-04 pb
%        WJets&  WJetsToLNu_TuneCUETP8M1_13TeV-madgraphMLM-pythia8       &5.030\times10^{+04} &  \\% 3.991e+01 pb
         \hline
        \end{tabular}
    }
    \caption{Table to test captions and labels}
    \label{table:1}
    \end{table}   
\end{frame}


\begin{frame}{Samples and Conditions}
    \begin{table}[htbp] %[h]
    \centering
%    \small
    \footnotesize%Use \footnotesize for a 20% (linear) reduction in font size
    \setlength\tabcolsep{2pt}%Reduce the amount of intercolumn whitespace
%    \begin{tabular}{|l | l | c|}
    \resizebox{\textwidth}{!}{% 
        \begin{tabular}{|l | l |>{$}c<{$}|c |} 
         \hline
         Process & Sample Name & Cross Section[pb] & notes\\% [0.5ex] %\[1ex]This adds extra space to the cell
         \hline
         \hline
        WZG    &  WZG_TuneCUETP8M1_13TeV-amcatnlo-pythia8                             & 4.123\times10^{-2}&  \\      %4.123e-02 +- 4.251e-05 pb
        WGG    &  WGG_5f_TuneCUETP8M1_13TeV-amcatnlo-pythia8                          & 1.819\times10^{0}&  \\      %1.819e+00 +- 5.227e-05 pb
%       WGG    &  WGGJets_TuneCUETP8M1_13TeV_madgraphMLM_pythia8                      & 1.711\times10^{0}&  \\      %1.711e+00 +- 1.753e-03 pb
        ZGG    &  ZGGJets_ZToHadOrNu_5f_LO_madgraph_pythia8                           & 3.717\times10^{-1}&  \\      %3.717e-01 +- 4.788e-04 pb
        WJets  &  WJetsToLNu_TuneCUETP8M1_13TeV-madgraphMLM-pythia8                   &5.030\times10^{+4} & ntuple in production  \\% 3.991e+01 pb
        DY     &  DYJetsToTauTau_ForcedMuEleDecay_M-50_TuneCUETP8M1_13TeV-amcatnloFXFX-pythia8_ext1 &1.983\times10^{+3} &  \\   %%1.983e+03 +- 4.355e+00 pb
        \hline
        single &  tZq_ll_4f_ckm_NLO_TuneCP5_PSweights_13TeV-amcatnlo-pythia8          &7.358\times10^{-2} &  \\   %%7.358e-02 +- 1.966e-04 pb
        top    &  tZq_nunu_4f_13TeV-amcatnlo-pythia8_TuneCUETP8M1                     &0000       &  \\   %
               &  ST_tW_antitop_5f_inclusiveDecays_13TeV-powheg-pythia8_TuneCUETP8M2T4&3.806\times10^{+1} &  \\   %%3.806e+01 +- 3.055e-02 pb
               &  ST_tW_top_5f_inclusiveDecays_13TeV-powheg-pythia8_TuneCUETP8M2T4    &3.809\times10^{+1} &  \\   %%3.809e+01 +- 3.050e-02 pb
        \hline
         TG    &  TGJets_TuneCUETP8M1_13TeV_amcatnlo_madspin_pythia8                  &2.967\times10^{+0} &  \\   %%2.967e+00 +- 1.052e-02 pb
        \hline
        TH     &  THW_ctcvcp_HIncl_M125_TuneCP5_13TeV-madgraph-pythia8                &1.467\times10^{-1} &  \\   %%1.467e-01 +- 1.485e-05 pb
               &  THQ_ctcvcp_Hincl_13TeV-madgraph-pythia8_TuneCUETP8M1                &8.816\times10^{-1} &  \\   %%8.816e-01 +- 1.991e-04 pb
        Z/W+H  &  VHToNonbb_M125_13TeV_amcatnloFXFX_madspin_pythia8                   &2.137\times10^{+0} &  \\   %%2.137e+00 +- 5.124e-03 pb
               &  ZHToTauTau_M125_13TeV_powheg_pythia8                                &7.524\times10^{-1} &  \\   %%7.524e-01 +- 3.643e-03 pb
               &  ZH_HToBB_ZToLL_M125_13TeV_powheg_pythia8                            &7.523\times10^{-2} &  \\   %%7.523e-02 +- 3.138e-04 pb
        ggFH   &  GluGluHToZZTo4L_M125_13TeV_powheg2_JHUgenV6_pythia8                 &2.999\times10^{+1} &  \\   %%2.999e+01 +- 2.112e-02 pb
               &  GluGluHToBB_M125_13TeV_amcatnloFXFX_pythia8                         &3.210\times10^{+1} &  \\   %%3.210e+01 +- 9.353e-02 pb
               &  GluGluHToGG_M125_13TeV_amcatnloFXFX_pythia8                         &3.198\times10^{+1} &  \\   %%3.198e+01 +- 9.594e-02 pb
               &  GluGluHToMuMu_M-125_TuneCP5_PSweights_13TeV_powheg_pythia8          &2.999\times10^{+1} &  \\   %%2.999e+01 +- 2.112e-02 pb
               &  GluGluHToTauTau_M125_13TeV_powheg_pythia8                           &3.052\times10^{+1} &  \\   %%3.052e+01 +- 2.150e-02 pb
               &  GluGluHToWWTo2L2Nu_M125_13TeV_powheg_JHUgen_pythia8                 &3.052\times10^{+1} &  \\   %%3.052e+01 +- 2.150e-02 pb
               &  GluGluHToWWToLNuQQ_M125_13TeV_powheg_JHUGenV628_pythia8             &2.999\times10^{+1} &  \\   %%2.999e+01 +- 2.112e-02 pb
%        VBFH  &  VBFHToWWToLNuQQ_M125_13TeV_powheg_JHUGenV628_pythia8   &3.769\times10^{+0} &  \\   %%3.769e+00 +- 1.589e-02 pb
%              &  VBFHToWWTo2L2Nu_M125_13TeV_powheg_JHUgenv628_pythia8   &3.769\times10^{+0} &  \\   %%3.769e+00 +- 1.589e-02 pb
%              &  VBFHToTauTau_M125_13TeV_powheg_pythia8   &  \\   %???3.721e+00 +/- 1.598e-02  final cross section = 0.000e+00 +-%???3.721e+00 +/- 1.598e-02  final cross section = 0.000e+00 +- 0.000e+00 pb
%          %???how can it the fianal cross section be 0?
%              &  VBFHToMuMu_M-125_TuneCP5_PSweights_13TeV_powheg_pythia8   &  \\   %???
%              &  VBFHToGG_M125_13TeV_amcatnlo_pythia8_v2   &3.992\times10^{+0} &  \\   %%3.992e+00 +- 8.933e-03 pb
%              &  VBFHToBB_M-125_13TeV_powheg_pythia8_weightfix   &  \\   %???
%              &  VBF_HToZZTo4L_M125_13TeV_powheg2_JHUgenV6_pythia8   &3.769\times10^{+0} &  \\   %%3.769e+00 +- 1.589e-02 pb
         \hline
        \end{tabular}
    }
    \caption{Table to test captions and labels}
    \label{table:2}
    \end{table}   
\end{frame}


\begin{frame}{Samples and Conditions}
    \begin{table}[htbp] %[h]
    \centering
%    \small
    \footnotesize%Use \footnotesize for a 20% (linear) reduction in font size
    \setlength\tabcolsep{2pt}%Reduce the amount of intercolumn whitespace
    \resizebox{\textwidth}{!}{% 
        \begin{tabular}{|l | l |>{$}c<{$}|} 
         \hline
         Process & Sample Name & Cross Section[pb]  \\% [0.5ex] %\[1ex]This adds extra space to the cell
         \hline
         \hline
        VBFH  &  VBFHToWWToLNuQQ_M125_13TeV_powheg_JHUGenV628_pythia8     &3.769\times10^{+0} \\   %%3.769e+00 +- 1.589e-02 pb
              &  VBFHToWWTo2L2Nu_M125_13TeV_powheg_JHUgenv628_pythia8     &3.769\times10^{+0} \\   %%3.769e+00 +- 1.589e-02 pb
              &  VBFHToTauTau_M125_13TeV_powheg_pythia8                   &  \\   %???3.721e+00 +/- 1.598e-02  final cross section = 0.000e+00 +-%???3.721e+00 +/- 1.598e-02  final cross section = 0.000e+00 +- 0.000e+00 pb
          %???how can it the fianal cross section be 0?
              &  VBFHToMuMu_M-125_TuneCP5_PSweights_13TeV_powheg_pythia8  &                 \\   %???
              &  VBFHToGG_M125_13TeV_amcatnlo_pythia8_v2                  &3.992\times10^{+0} \\   %%3.992e+00 +- 8.933e-03 pb
              &  VBFHToBB_M-125_13TeV_powheg_pythia8_weightfix            &         \\   %???
              &  VBF_HToZZTo4L_M125_13TeV_powheg2_JHUgenV6_pythia8        &3.769\times10^{+0} \\   %%3.769e+00 +- 1.589e-02 pb
         \hline
        \end{tabular}
    }
    \caption{Table to test captions and labels}
    \label{table:2}
    \end{table}   
\end{frame}




\subsection{Object Selection}


\begin{frame}{Object Selection}
  \begin{itemize}
  \item
    Electron 
    \begin{itemize}
    \item
        pt>20, |eta|<2.4
    \item
        electron cut based loose ID(Fall-94X-V2)
    \end{itemize}
  \item
    Muon
    \begin{itemize}
    \item
        pt>20, |eta|<2.4
    \item
        cutbased ID (recommended by muon POG)
    \item
        pass loose isolation(same as SS of TTTT)
    \end{itemize}
  \item
    Tau
    \begin{itemize}
    \item
        pt>20, |eta|<2.3
    \item
        tau ID(same as ttH)
    \end{itemize}
  \end{itemize}
\end{frame}


\begin{frame}{Object Selection}
  \begin{itemize}
  \item
    Jet
    \begin{itemize}
    \item
        pt>25
    \item
        loose jet(recommended by JETMET)
    \end{itemize}
  \item
     B Jet
     \begin{itemize}
        \item
            use Deep Flavour B tagging algorithm 
        \item 
            use the recommended working points
     \end{itemize}
  \item
      Top 
    \begin{itemize}
    \item
        use SUSY HOT TopTagger 
    \item 
        resolved
    \end{itemize}

  \end{itemize}
\end{frame}


\begin{frame}{Event Selection-Preselection}
    \begin{itemize}
    \item
        MET filters
    \begin{table}[htbp] %[h]
    \centering
%    \small
    \footnotesize%Use \footnotesize for a 20% (linear) reduction in font size
%    \setlength\tabcolsep{2pt}%Reduce the amount of intercolumn whitespace
    \resizebox{0.8\textwidth}{!}{% 
        \begin{tabular}{|l |>{$}c<{$}|>{$}c<{$}|} 
         \hline
         filter & applied to MC & applied to data  \\% [0.5ex] %\[1ex]This adds extra space to the cell
         \hline
         \hline
         Flag_goodVertices                       & \surd   & \surd  \\
         Flag_globalSuperTightHalo2016Filter     & \surd   &   \surd\\
         Flag_HBHENoiseFilter                    & \surd   & \surd \\ 
         Flag_HBHENoiseIsoFilter                 & \surd   & \surd \\
         Flag_EcalDeadCellTriggerPrimitiveFilter & \surd & \surd \\
         Flag_BadPFMuonFilter                    & \surd & \surd \\
         Flag_eeBadScFilter                      & \times  & \surd  \\
         \hline
        \end{tabular}
    }
    \caption{MET Filters}
    \label{table:3}
    \end{table}   
    \item
        at least 1 loose tau
    \item
        at least 1 loose jet 
    \item
        at least 1 loose b jet
    \end{itemize}
\end{frame}


\begin{frame}{Subchannel Categorization}
    \begin{itemize}
    \item
        categorization a bit diffrent than ttH
    \item
        we have medium tau for each channel while they do not.
    \item
        need to optimaze channel categorization
    \end{itemize}
    \begin{table}[htbp] %[h]
    \centering
%    \small
    \footnotesize%Use \footnotesize for a 20% (linear) reduction in font size
    \setlength\tabcolsep{2pt}%Reduce the amount of intercolumn whitespace
%    \begin{tabular}{|l | l | c|}
    \resizebox{0.8\textwidth}{!}{% 
%        \begin{tabular}{|l | l |>{$}c<{$}|} 
        \begin{tabular}{|c | c |c|} 
         \hline
         channel & lepton & tau  \\% [0.5ex] %\[1ex]This adds extra space to the cell
         \hline
         \hline
         1Tau 0L & 0 tight electrons or muons & 1 medium tau\\  
         1Tau 1L & exact 1 tight electron or 1 tight muon & 1 medium tau \\ 
         1Tau 2OSL & 2 tight leptons of the opposite charge & 1 medium tau\\
         1Tau 2SSL & 2 tight leptons of the same charge & 1 medium tau\\
         1Tau 3L & 3 tight letons & 1 medium tau\\
         2Tau 0L & 0 tight electrons or muons& 2 medium tau\\
         2Tau 1L & exact 1 tight electron or 1 tighy moun & 2 medium tau\\
         2Tau 2OSL&2 tight leptons of the opposite charge & 2 medim tau\\
         2Tau 2SSL & 2 tight leptons of the same charge & 2 medium tau\\
%         2Tau 3L & 3 tight lepton & 2 medium tau\\
         \hline
        \end{tabular}
    }
    \caption{Subchannel difinition}
    \label{table:3}
    \end{table}   
\end{frame}


\begin{frame}{1Tau0L}
    \begin{itemize}
    \item
        background is all the samples except W + Jets
    \item
        normarlized to cross section
    \end{itemize}
    \begin{columns}[t]
    \column{.5\textwidth}
    \centering
    \includegraphics[width=4.1cm]{HTDividedByMET1Tau0L.png}\\
%    \includegraphics[width=4.0cm]{LeadingTopPt1Tau0L.png}
%    \caption{background is all the samples excerpt W + Jets. normarlized to cross section}
    \column{.5\textwidth}
    \centering
    \includegraphics[width=4.1cm]{InvariantMassJets1Tau0L.png}
    \end{columns}
\end{frame}
\begin{frame}{1Tau0L}
    \begin{columns}[t]
    \column{.5\textwidth}
    \centering
    \includegraphics[width=4.1cm]{LeadingBJetPt1Tau0L.png}\\
    \includegraphics[width=4.1cm]{SecondBJetPt1Tau0L.png}
    \column{.5\textwidth}
    \centering
    \includegraphics[width=4.1cm]{ThirdBJetPt1Tau0L.png}\\
    \includegraphics[width=4.1cm]{LeadingTauPt1Tau0L.png}
    \end{columns}
\end{frame}
\begin{frame}{1Tau0L}
    \begin{columns}[t]
    \column{.5\textwidth}
    \centering
    \includegraphics[width=4.1cm]{NumSelJets1Tau0L.png}\\
    \includegraphics[width=4.1cm]{LeadingJetPt1Tau0L.png}
    \column{.5\textwidth}
    \centering
    \includegraphics[width=4.1cm]{SecondJetPt1Tau0L.png}\\
    \includegraphics[width=4.1cm]{ThirdJetPt1Tau0L.png}
    \end{columns}
\end{frame}
\begin{frame}{1Tau0L}
    \begin{columns}[t]
    \column{.5\textwidth}
    \centering
    \includegraphics[width=4.1cm]{FourthJetPt1Tau0L.png}\\
    \includegraphics[width=4.1cm]{FifthJetPt1Tau0L.png}
    \column{.5\textwidth}
    \centering
    \includegraphics[width=4.1cm]{SixthJetPt1Tau0L.png}\\
    \includegraphics[width=4.1cm]{Centrality1Tau0L.png}
    \end{columns}
\end{frame}
\begin{frame}{1Tau0L}
    \begin{columns}[t]
    \column{.5\textwidth}
    \centering
    \includegraphics[width=4.1cm]{HT1Tau0L.png}\\
    \includegraphics[width=4.1cm]{MHTDividedByMET1Tau0L.png}
    \column{.5\textwidth}
    \centering
    \includegraphics[width=4.1cm]{MaxDeltaRJets1Tau0L.png}\\
    \includegraphics[width=4.1cm]{MinDeltaRJets1Tau0L.png}
    \end{columns}
\end{frame}
\begin{frame}{1Tau0L}
    \begin{columns}[t]
    \column{.5\textwidth}
    \centering
    \includegraphics[width=4.1cm]{SecondTopPt1Tau0L.png}\\
    \includegraphics[width=4.1cm]{NumSelBJetsM1Tau0L.png}
    \column{.5\textwidth}
    \centering
    \includegraphics[width=4.1cm]{NumofTops1Tau0L.png}\\
    \includegraphics[width=4.1cm]{LeadingTopPt1Tau0L.png}
%    \includegraphics[width=4.1cm]{Met_pt1Tau0L.png}\\%Met_pt1Tau0L.png
%    \includegraphics[width=4.1cm]{1Tau0L.png}
    \end{columns}
\end{frame}


\begin{frame}{1Tau1L}
    \begin{columns}[t]
    \column{.5\textwidth}
    \centering
    \includegraphics[width=4.1cm]{HTDividedByMET1Tau1L.png}\\
%    \includegraphics[width=4.0cm]{LeadingTopPt1Tau1L.png}
%    \caption{background is all the samples excerpt W + Jets. normarlized to cross section}
    \column{.5\textwidth}
    \centering
    \includegraphics[width=4.1cm]{InvariantMassJets1Tau1L.png}
    \end{columns}
\end{frame}
\begin{frame}{1Tau1L}
    \begin{columns}[t]
    \column{.5\textwidth}
    \centering
    \includegraphics[width=4.1cm]{LeadingBJetPt1Tau1L.png}\\
    \includegraphics[width=4.1cm]{SecondBJetPt1Tau1L.png}
    \column{.5\textwidth}
    \centering
    \includegraphics[width=4.1cm]{ThirdBJetPt1Tau1L.png}\\
    \includegraphics[width=4.1cm]{LeadingTauPt1Tau1L.png}
    \end{columns}
\end{frame}
\begin{frame}{1Tau1L}
    \begin{columns}[t]
    \column{.5\textwidth}
    \centering
    \includegraphics[width=4.1cm]{NumSelJets1Tau1L.png}\\
    \includegraphics[width=4.1cm]{LeadingJetPt1Tau1L.png}
    \column{.5\textwidth}
    \centering
    \includegraphics[width=4.1cm]{SecondJetPt1Tau1L.png}\\
    \includegraphics[width=4.1cm]{ThirdJetPt1Tau1L.png}
    \end{columns}
\end{frame}
\begin{frame}{1Tau1L}
    \begin{columns}[t]
    \column{.5\textwidth}
    \centering
    \includegraphics[width=4.1cm]{FourthJetPt1Tau1L.png}\\
    \includegraphics[width=4.1cm]{FifthJetPt1Tau1L.png}
    \column{.5\textwidth}
    \centering
    \includegraphics[width=4.1cm]{SixthJetPt1Tau1L.png}\\
    \includegraphics[width=4.1cm]{Centrality1Tau1L.png}
    \end{columns}
\end{frame}
\begin{frame}{1Tau1L}
    \begin{columns}[t]
    \column{.5\textwidth}
    \centering
    \includegraphics[width=4.1cm]{HT1Tau1L.png}\\
    \includegraphics[width=4.1cm]{MHTDividedByMET1Tau1L.png}
    \column{.5\textwidth}
    \centering
    \includegraphics[width=4.1cm]{MaxDeltaRJets1Tau1L.png}\\
    \includegraphics[width=4.1cm]{MinDeltaRJets1Tau1L.png}
    \end{columns}
\end{frame}
\begin{frame}{1Tau1L}
    \begin{columns}[t]
    \column{.5\textwidth}
    \centering
    \includegraphics[width=4.1cm]{SecondTopPt1Tau1L.png}\\
    \includegraphics[width=4.1cm]{NumSelBJetsM1Tau1L.png}
    \column{.5\textwidth}
    \centering
    \includegraphics[width=4.1cm]{NumofTops1Tau1L.png}\\
    \includegraphics[width=4.1cm]{LeadingTopPt1Tau1L.png}
%    \includegraphics[width=4.1cm]{Met_pt1Tau1L.png}\\%Met_pt1Tau1L.png
%    \includegraphics[width=4.1cm]{1Tau1L.png}
    \end{columns}
\end{frame}



\begin{frame}{1Tau2OS}
    \begin{columns}[t]
    \column{.5\textwidth}
    \centering
    \includegraphics[width=4.1cm]{HTDividedByMET1Tau2OS.png}\\
%    \includegraphics[width=4.0cm]{LeadingTopPt1Tau2OS.png}
%    \caption{background is all the samples excerpt W + Jets. normarlized to cross section}
    \column{.5\textwidth}
    \centering
    \includegraphics[width=4.1cm]{InvariantMassJets1Tau2OS.png}
    \end{columns}
\end{frame}
\begin{frame}{1Tau2OS}
    \begin{columns}[t]
    \column{.5\textwidth}
    \centering
    \includegraphics[width=4.1cm]{LeadingBJetPt1Tau2OS.png}\\
    \includegraphics[width=4.1cm]{SecondBJetPt1Tau2OS.png}
    \column{.5\textwidth}
    \centering
    \includegraphics[width=4.1cm]{ThirdBJetPt1Tau2OS.png}\\
    \includegraphics[width=4.1cm]{LeadingTauPt1Tau2OS.png}
    \end{columns}
\end{frame}
\begin{frame}{1Tau2OS}
    \begin{columns}[t]
    \column{.5\textwidth}
    \centering
    \includegraphics[width=4.1cm]{NumSelJets1Tau2OS.png}\\
    \includegraphics[width=4.1cm]{LeadingJetPt1Tau2OS.png}
    \column{.5\textwidth}
    \centering
    \includegraphics[width=4.1cm]{SecondJetPt1Tau2OS.png}\\
    \includegraphics[width=4.1cm]{ThirdJetPt1Tau2OS.png}
    \end{columns}
\end{frame}
\begin{frame}{1Tau2OS}
    \begin{columns}[t]
    \column{.5\textwidth}
    \centering
    \includegraphics[width=4.1cm]{FourthJetPt1Tau2OS.png}\\
    \includegraphics[width=4.1cm]{FifthJetPt1Tau2OS.png}
    \column{.5\textwidth}
    \centering
    \includegraphics[width=4.1cm]{SixthJetPt1Tau2OS.png}\\
    \includegraphics[width=4.1cm]{Centrality1Tau2OS.png}
    \end{columns}
\end{frame}
\begin{frame}{1Tau2OS}
    \begin{columns}[t]
    \column{.5\textwidth}
    \centering
    \includegraphics[width=4.1cm]{HT1Tau2OS.png}\\
    \includegraphics[width=4.1cm]{MHTDividedByMET1Tau2OS.png}
    \column{.5\textwidth}
    \centering
    \includegraphics[width=4.1cm]{MaxDeltaRJets1Tau2OS.png}\\
    \includegraphics[width=4.1cm]{MinDeltaRJets1Tau2OS.png}
    \end{columns}
\end{frame}
\begin{frame}{1Tau2OS}
    \begin{columns}[t]
    \column{.5\textwidth}
    \centering
    \includegraphics[width=4.1cm]{SecondTopPt1Tau2OS.png}\\
    \includegraphics[width=4.1cm]{NumSelBJetsM1Tau2OS.png}
    \column{.5\textwidth}
    \centering
    \includegraphics[width=4.1cm]{NumofTops1Tau2OS.png}\\
    \includegraphics[width=4.1cm]{LeadingTopPt1Tau2OS.png}
%    \includegraphics[width=4.1cm]{Met_pt1Tau2OS.png}\\%Met_pt1Tau2OS.png
%    \includegraphics[width=4.1cm]{1Tau2OS.png}
    \end{columns}
\end{frame}



\begin{frame}{1Tau2SS}
    \begin{columns}[t]
    \column{.5\textwidth}
    \centering
    \includegraphics[width=4.1cm]{HTDividedByMET1Tau2SS.png}\\
%    \includegraphics[width=4.0cm]{LeadingTopPt1Tau2SS.png}
%    \caption{background is all the samples excerpt W + Jets. normarlized to cross section}
    \column{.5\textwidth}
    \centering
    \includegraphics[width=4.1cm]{InvariantMassJets1Tau2SS.png}
    \end{columns}
\end{frame}
\begin{frame}{1Tau2SS}
    \begin{columns}[t]
    \column{.5\textwidth}
    \centering
    \includegraphics[width=4.1cm]{LeadingBJetPt1Tau2SS.png}\\
    \includegraphics[width=4.1cm]{SecondBJetPt1Tau2SS.png}
    \column{.5\textwidth}
    \centering
    \includegraphics[width=4.1cm]{ThirdBJetPt1Tau2SS.png}\\
    \includegraphics[width=4.1cm]{LeadingTauPt1Tau2SS.png}
    \end{columns}
\end{frame}
\begin{frame}{1Tau2SS}
    \begin{columns}[t]
    \column{.5\textwidth}
    \centering
    \includegraphics[width=4.1cm]{NumSelJets1Tau2SS.png}\\
    \includegraphics[width=4.1cm]{LeadingJetPt1Tau2SS.png}
    \column{.5\textwidth}
    \centering
    \includegraphics[width=4.1cm]{SecondJetPt1Tau2SS.png}\\
    \includegraphics[width=4.1cm]{ThirdJetPt1Tau2SS.png}
    \end{columns}
\end{frame}
\begin{frame}{1Tau2SS}
    \begin{columns}[t]
    \column{.5\textwidth}
    \centering
    \includegraphics[width=4.1cm]{FourthJetPt1Tau2SS.png}\\
    \includegraphics[width=4.1cm]{FifthJetPt1Tau2SS.png}
    \column{.5\textwidth}
    \centering
    \includegraphics[width=4.1cm]{SixthJetPt1Tau2SS.png}\\
    \includegraphics[width=4.1cm]{Centrality1Tau2SS.png}
    \end{columns}
\end{frame}
\begin{frame}{1Tau2SS}
    \begin{columns}[t]
    \column{.5\textwidth}
    \centering
    \includegraphics[width=4.1cm]{HT1Tau2SS.png}\\
    \includegraphics[width=4.1cm]{MHTDividedByMET1Tau2SS.png}
    \column{.5\textwidth}
    \centering
    \includegraphics[width=4.1cm]{MaxDeltaRJets1Tau2SS.png}\\
    \includegraphics[width=4.1cm]{MinDeltaRJets1Tau2SS.png}
    \end{columns}
\end{frame}
\begin{frame}{1Tau2SS}
    \begin{columns}[t]
    \column{.5\textwidth}
    \centering
    \includegraphics[width=4.1cm]{SecondTopPt1Tau2SS.png}\\
    \includegraphics[width=4.1cm]{NumSelBJetsM1Tau2SS.png}
    \column{.5\textwidth}
    \centering
    \includegraphics[width=4.1cm]{NumofTops1Tau2SS.png}\\
    \includegraphics[width=4.1cm]{LeadingTopPt1Tau2SS.png}
%    \includegraphics[width=4.1cm]{Met_pt1Tau2SS.png}\\%Met_pt1Tau2OS.png
%    \includegraphics[width=4.1cm]{1Tau2SS.png}
    \end{columns}
\end{frame}


\begin{frame}{1Tau2SS}
    \begin{columns}[t]
    \column{.5\textwidth}
    \centering
    \includegraphics[width=4.1cm]{HTDividedByMET1Tau2SS.png}\\
%    \includegraphics[width=4.0cm]{LeadingTopPt1Tau2SS.png}
%    \caption{background is all the samples excerpt W + Jets. normarlized to cross section}
    \column{.5\textwidth}
    \centering
    \includegraphics[width=4.1cm]{InvariantMassJets1Tau2SS.png}
    \end{columns}
\end{frame}
\begin{frame}{1Tau2SS}
    \begin{columns}[t]
    \column{.5\textwidth}
    \centering
    \includegraphics[width=4.1cm]{LeadingBJetPt1Tau2SS.png}\\
    \includegraphics[width=4.1cm]{SecondBJetPt1Tau2SS.png}
    \column{.5\textwidth}
    \centering
    \includegraphics[width=4.1cm]{ThirdBJetPt1Tau2SS.png}\\
    \includegraphics[width=4.1cm]{LeadingTauPt1Tau2SS.png}
    \end{columns}
\end{frame}
\begin{frame}{1Tau2SS}
    \begin{columns}[t]
    \column{.5\textwidth}
    \centering
    \includegraphics[width=4.1cm]{NumSelJets1Tau2SS.png}\\
    \includegraphics[width=4.1cm]{LeadingJetPt1Tau2SS.png}
    \column{.5\textwidth}
    \centering
    \includegraphics[width=4.1cm]{SecondJetPt1Tau2SS.png}\\
    \includegraphics[width=4.1cm]{ThirdJetPt1Tau2SS.png}
    \end{columns}
\end{frame}
\begin{frame}{1Tau2SS}
    \begin{columns}[t]
    \column{.5\textwidth}
    \centering
    \includegraphics[width=4.1cm]{FourthJetPt1Tau2SS.png}\\
    \includegraphics[width=4.1cm]{FifthJetPt1Tau2SS.png}
    \column{.5\textwidth}
    \centering
    \includegraphics[width=4.1cm]{SixthJetPt1Tau2SS.png}\\
    \includegraphics[width=4.1cm]{Centrality1Tau2SS.png}
    \end{columns}
\end{frame}
\begin{frame}{1Tau2SS}
    \begin{columns}[t]
    \column{.5\textwidth}
    \centering
    \includegraphics[width=4.1cm]{HT1Tau2SS.png}\\
    \includegraphics[width=4.1cm]{MHTDividedByMET1Tau2SS.png}
    \column{.5\textwidth}
    \centering
    \includegraphics[width=4.1cm]{MaxDeltaRJets1Tau2SS.png}\\
    \includegraphics[width=4.1cm]{MinDeltaRJets1Tau2SS.png}
    \end{columns}
\end{frame}
\begin{frame}{1Tau2SS}
    \begin{columns}[t]
    \column{.5\textwidth}
    \centering
    \includegraphics[width=4.1cm]{SecondTopPt1Tau2SS.png}\\
    \includegraphics[width=4.1cm]{NumSelBJetsM1Tau2SS.png}
    \column{.5\textwidth}
    \centering
    \includegraphics[width=4.1cm]{NumofTops1Tau2SS.png}\\
    \includegraphics[width=4.1cm]{LeadingTopPt1Tau2SS.png}
%    \includegraphics[width=4.1cm]{Met_pt1Tau2SS.png}\\%Met_pt1Tau2OS.png
%    \includegraphics[width=4.1cm]{1Tau2SS.png}
    \end{columns}
\end{frame}


\begin{frame}{1Tau2SS}
    \begin{columns}[t]
    \column{.5\textwidth}
    \centering
    \includegraphics[width=4.1cm]{HTDividedByMET1Tau2SS.png}\\
%    \includegraphics[width=4.0cm]{LeadingTopPt1Tau2SS.png}
%    \caption{background is all the samples excerpt W + Jets. normarlized to cross section}
    \column{.5\textwidth}
    \centering
    \includegraphics[width=4.1cm]{InvariantMassJets1Tau2SS.png}
    \end{columns}
\end{frame}
\begin{frame}{1Tau2SS}
    \begin{columns}[t]
    \column{.5\textwidth}
    \centering
    \includegraphics[width=4.1cm]{LeadingBJetPt1Tau2SS.png}\\
    \includegraphics[width=4.1cm]{SecondBJetPt1Tau2SS.png}
    \column{.5\textwidth}
    \centering
    \includegraphics[width=4.1cm]{ThirdBJetPt1Tau2SS.png}\\
    \includegraphics[width=4.1cm]{LeadingTauPt1Tau2SS.png}
    \end{columns}
\end{frame}
\begin{frame}{1Tau2SS}
    \begin{columns}[t]
    \column{.5\textwidth}
    \centering
    \includegraphics[width=4.1cm]{NumSelJets1Tau2SS.png}\\
    \includegraphics[width=4.1cm]{LeadingJetPt1Tau2SS.png}
    \column{.5\textwidth}
    \centering
    \includegraphics[width=4.1cm]{SecondJetPt1Tau2SS.png}\\
    \includegraphics[width=4.1cm]{ThirdJetPt1Tau2SS.png}
    \end{columns}
\end{frame}
\begin{frame}{1Tau2SS}
    \begin{columns}[t]
    \column{.5\textwidth}
    \centering
    \includegraphics[width=4.1cm]{FourthJetPt1Tau2SS.png}\\
    \includegraphics[width=4.1cm]{FifthJetPt1Tau2SS.png}
    \column{.5\textwidth}
    \centering
    \includegraphics[width=4.1cm]{SixthJetPt1Tau2SS.png}\\
    \includegraphics[width=4.1cm]{Centrality1Tau2SS.png}
    \end{columns}
\end{frame}
\begin{frame}{1Tau2SS}
    \begin{columns}[t]
    \column{.5\textwidth}
    \centering
    \includegraphics[width=4.1cm]{HT1Tau2SS.png}\\
    \includegraphics[width=4.1cm]{MHTDividedByMET1Tau2SS.png}
    \column{.5\textwidth}
    \centering
    \includegraphics[width=4.1cm]{MaxDeltaRJets1Tau2SS.png}\\
    \includegraphics[width=4.1cm]{MinDeltaRJets1Tau2SS.png}
    \end{columns}
\end{frame}
\begin{frame}{1Tau2SS}
    \begin{columns}[t]
    \column{.5\textwidth}
    \centering
    \includegraphics[width=4.1cm]{SecondTopPt1Tau2SS.png}\\
    \includegraphics[width=4.1cm]{NumSelBJetsM1Tau2SS.png}
    \column{.5\textwidth}
    \centering
    \includegraphics[width=4.1cm]{NumofTops1Tau2SS.png}\\
    \includegraphics[width=4.1cm]{LeadingTopPt1Tau2SS.png}
%    \includegraphics[width=4.1cm]{Met_pt1Tau2SS.png}\\%Met_pt1Tau2OS.png
%    \includegraphics[width=4.1cm]{1Tau2SS.png}
    \end{columns}
\end{frame}


\begin{frame}{2Tau0L}
    \begin{columns}[t]
    \column{.5\textwidth}
    \centering
    \includegraphics[width=4.1cm]{HTDividedByMET2Tau0L.png}\\
%    \includegraphics[width=4.0cm]{LeadingTopPt2Tau0L.png}
%    \caption{background is all the samples excerpt W + Jets. normarlized to cross section}
    \column{.5\textwidth}
    \centering
    \includegraphics[width=4.1cm]{InvariantMassJets2Tau0L.png}
    \end{columns}
\end{frame}
\begin{frame}{2Tau0L}
    \begin{columns}[t]
    \column{.5\textwidth}
    \centering
    \includegraphics[width=4.1cm]{LeadingBJetPt2Tau0L.png}\\
    \includegraphics[width=4.1cm]{SecondBJetPt2Tau0L.png}
    \column{.5\textwidth}
    \centering
    \includegraphics[width=4.1cm]{ThirdBJetPt2Tau0L.png}\\
    \includegraphics[width=4.1cm]{LeadingTauPt2Tau0L.png}
    \end{columns}
\end{frame}
\begin{frame}{2Tau0L}
    \begin{columns}[t]
    \column{.5\textwidth}
    \centering
    \includegraphics[width=4.1cm]{NumSelJets2Tau0L.png}\\
    \includegraphics[width=4.1cm]{LeadingJetPt2Tau0L.png}
    \column{.5\textwidth}
    \centering
    \includegraphics[width=4.1cm]{SecondJetPt2Tau0L.png}\\
    \includegraphics[width=4.1cm]{ThirdJetPt2Tau0L.png}
    \end{columns}
\end{frame}
\begin{frame}{2Tau0L}
    \begin{columns}[t]
    \column{.5\textwidth}
    \centering
    \includegraphics[width=4.1cm]{FourthJetPt2Tau0L.png}\\
    \includegraphics[width=4.1cm]{FifthJetPt2Tau0L.png}
    \column{.5\textwidth}
    \centering
    \includegraphics[width=4.1cm]{SixthJetPt2Tau0L.png}\\
    \includegraphics[width=4.1cm]{Centrality2Tau0L.png}
    \end{columns}
\end{frame}
\begin{frame}{2Tau0L}
    \begin{columns}[t]
    \column{.5\textwidth}
    \centering
    \includegraphics[width=4.1cm]{HT2Tau0L.png}\\
    \includegraphics[width=4.1cm]{MHTDividedByMET2Tau0L.png}
    \column{.5\textwidth}
    \centering
    \includegraphics[width=4.1cm]{MaxDeltaRJets2Tau0L.png}\\
    \includegraphics[width=4.1cm]{MinDeltaRJets2Tau0L.png}
    \end{columns}
\end{frame}
\begin{frame}{2Tau0L}
    \begin{columns}[t]
    \column{.5\textwidth}
    \centering
    \includegraphics[width=4.1cm]{SecondTopPt2Tau0L.png}\\
    \includegraphics[width=4.1cm]{NumSelBJetsM2Tau0L.png}
    \column{.5\textwidth}
    \centering
    \includegraphics[width=4.1cm]{NumofTops2Tau0L.png}\\
    \includegraphics[width=4.1cm]{LeadingTopPt2Tau0L.png}
%    \includegraphics[width=4.1cm]{Met_pt2Tau0L.png}\\%Met_pt1Tau2OS.png
%    \includegraphics[width=4.1cm]{2Tau0L.png}
    \end{columns}
\end{frame}


\begin{frame}{2Tau1L}
    \begin{columns}[t]
    \column{.5\textwidth}
    \centering
    \includegraphics[width=4.1cm]{HTDividedByMET2Tau1L.png}\\
%    \includegraphics[width=4.0cm]{LeadingTopPt2Tau1L.png}
%    \caption{background is all the samples excerpt W + Jets. normarlized to cross section}
    \column{.5\textwidth}
    \centering
    \includegraphics[width=4.1cm]{InvariantMassJets2Tau1L.png}
    \end{columns}
\end{frame}
\begin{frame}{2Tau1L}
    \begin{columns}[t]
    \column{.5\textwidth}
    \centering
    \includegraphics[width=4.1cm]{LeadingBJetPt2Tau1L.png}\\
    \includegraphics[width=4.1cm]{SecondBJetPt2Tau1L.png}
    \column{.5\textwidth}
    \centering
    \includegraphics[width=4.1cm]{ThirdBJetPt2Tau1L.png}\\
    \includegraphics[width=4.1cm]{LeadingTauPt2Tau1L.png}
    \end{columns}
\end{frame}
\begin{frame}{2Tau1L}
    \begin{columns}[t]
    \column{.5\textwidth}
    \centering
    \includegraphics[width=4.1cm]{NumSelJets2Tau1L.png}\\
    \includegraphics[width=4.1cm]{LeadingJetPt2Tau1L.png}
    \column{.5\textwidth}
    \centering
    \includegraphics[width=4.1cm]{SecondJetPt2Tau1L.png}\\
    \includegraphics[width=4.1cm]{ThirdJetPt2Tau1L.png}
    \end{columns}
\end{frame}
\begin{frame}{2Tau1L}
    \begin{columns}[t]
    \column{.5\textwidth}
    \centering
    \includegraphics[width=4.1cm]{FourthJetPt2Tau1L.png}\\
    \includegraphics[width=4.1cm]{FifthJetPt2Tau1L.png}
    \column{.5\textwidth}
    \centering
    \includegraphics[width=4.1cm]{SixthJetPt2Tau1L.png}\\
    \includegraphics[width=4.1cm]{Centrality2Tau1L.png}
    \end{columns}
\end{frame}
\begin{frame}{2Tau1L}
    \begin{columns}[t]
    \column{.5\textwidth}
    \centering
    \includegraphics[width=4.1cm]{HT2Tau1L.png}\\
    \includegraphics[width=4.1cm]{MHTDividedByMET2Tau1L.png}
    \column{.5\textwidth}
    \centering
    \includegraphics[width=4.1cm]{MaxDeltaRJets2Tau1L.png}\\
    \includegraphics[width=4.1cm]{MinDeltaRJets2Tau1L.png}
    \end{columns}
\end{frame}
\begin{frame}{2Tau1L}
    \begin{columns}[t]
    \column{.5\textwidth}
    \centering
    \includegraphics[width=4.1cm]{SecondTopPt2Tau1L.png}\\
    \includegraphics[width=4.1cm]{NumSelBJetsM2Tau1L.png}
    \column{.5\textwidth}
    \centering
    \includegraphics[width=4.1cm]{NumofTops2Tau1L.png}\\
    \includegraphics[width=4.1cm]{LeadingTopPt2Tau1L.png}
%    \includegraphics[width=4.1cm]{Met_pt2Tau1L.png}\\%Met_pt1Tau2OS.png
%    \includegraphics[width=4.1cm]{2Tau1L.png}
    \end{columns}
\end{frame}



\begin{frame}{2Tau2OS}
    \begin{columns}[t]
    \column{.5\textwidth}
    \centering
    \includegraphics[width=4.1cm]{HTDividedByMET2Tau2OS.png}\\
%    \includegraphics[width=4.0cm]{LeadingTopPt2Tau2OS.png}
%    \caption{background is all the samples excerpt W + Jets. normarlized to cross section}
    \column{.5\textwidth}
    \centering
    \includegraphics[width=4.1cm]{InvariantMassJets2Tau2OS.png}
    \end{columns}
\end{frame}
\begin{frame}{2Tau2OS}
    \begin{columns}[t]
    \column{.5\textwidth}
    \centering
    \includegraphics[width=4.1cm]{LeadingBJetPt2Tau2OS.png}\\
    \includegraphics[width=4.1cm]{SecondBJetPt2Tau2OS.png}
    \column{.5\textwidth}
    \centering
    \includegraphics[width=4.1cm]{ThirdBJetPt2Tau2OS.png}\\
    \includegraphics[width=4.1cm]{LeadingTauPt2Tau2OS.png}
    \end{columns}
\end{frame}
\begin{frame}{2Tau2OS}
    \begin{columns}[t]
    \column{.5\textwidth}
    \centering
    \includegraphics[width=4.1cm]{NumSelJets2Tau2OS.png}\\
    \includegraphics[width=4.1cm]{LeadingJetPt2Tau2OS.png}
    \column{.5\textwidth}
    \centering
    \includegraphics[width=4.1cm]{SecondJetPt2Tau2OS.png}\\
    \includegraphics[width=4.1cm]{ThirdJetPt2Tau2OS.png}
    \end{columns}
\end{frame}
\begin{frame}{2Tau2OS}
    \begin{columns}[t]
    \column{.5\textwidth}
    \centering
    \includegraphics[width=4.1cm]{FourthJetPt2Tau2OS.png}\\
    \includegraphics[width=4.1cm]{FifthJetPt2Tau2OS.png}
    \column{.5\textwidth}
    \centering
    \includegraphics[width=4.1cm]{SixthJetPt2Tau2OS.png}\\
    \includegraphics[width=4.1cm]{Centrality2Tau2OS.png}
    \end{columns}
\end{frame}
\begin{frame}{2Tau2OS}
    \begin{columns}[t]
    \column{.5\textwidth}
    \centering
    \includegraphics[width=4.1cm]{HT2Tau2OS.png}\\
    \includegraphics[width=4.1cm]{MHTDividedByMET2Tau2OS.png}
    \column{.5\textwidth}
    \centering
    \includegraphics[width=4.1cm]{MaxDeltaRJets2Tau2OS.png}\\
    \includegraphics[width=4.1cm]{MinDeltaRJets2Tau2OS.png}
    \end{columns}
\end{frame}
\begin{frame}{2Tau2OS}
    \begin{columns}[t]
    \column{.5\textwidth}
    \centering
    \includegraphics[width=4.1cm]{SecondTopPt2Tau2OS.png}\\
    \includegraphics[width=4.1cm]{NumSelBJetsM2Tau2OS.png}
    \column{.5\textwidth}
    \centering
    \includegraphics[width=4.1cm]{NumofTops2Tau2OS.png}\\
    \includegraphics[width=4.1cm]{LeadingTopPt2Tau2OS.png}
%    \includegraphics[width=4.1cm]{Met_pt2Tau2OS.png}\\%Met_pt1Tau2OS.png
%    \includegraphics[width=4.1cm]{2Tau2OS.png}
    \end{columns}
\end{frame}



\begin{frame}{2Tau2SS}
    \begin{columns}[t]
    \column{.5\textwidth}
    \centering
    \includegraphics[width=4.1cm]{HTDividedByMET2Tau2SS.png}\\
%    \includegraphics[width=4.0cm]{LeadingTopPt2Tau2SS.png}
%    \caption{background is all the samples excerpt W + Jets. normarlized to cross section}
    \column{.5\textwidth}
    \centering
    \includegraphics[width=4.1cm]{InvariantMassJets2Tau2SS.png}
    \end{columns}
\end{frame}
\begin{frame}{2Tau2SS}
    \begin{columns}[t]
    \column{.5\textwidth}
    \centering
    \includegraphics[width=4.1cm]{LeadingBJetPt2Tau2SS.png}\\
    \includegraphics[width=4.1cm]{SecondBJetPt2Tau2SS.png}
    \column{.5\textwidth}
    \centering
    \includegraphics[width=4.1cm]{ThirdBJetPt2Tau2SS.png}\\
    \includegraphics[width=4.1cm]{LeadingTauPt2Tau2SS.png}
    \end{columns}
\end{frame}
\begin{frame}{2Tau2SS}
    \begin{columns}[t]
    \column{.5\textwidth}
    \centering
    \includegraphics[width=4.1cm]{NumSelJets2Tau2SS.png}\\
    \includegraphics[width=4.1cm]{LeadingJetPt2Tau2SS.png}
    \column{.5\textwidth}
    \centering
    \includegraphics[width=4.1cm]{SecondJetPt2Tau2SS.png}\\
    \includegraphics[width=4.1cm]{ThirdJetPt2Tau2SS.png}
    \end{columns}
\end{frame}
\begin{frame}{2Tau2SS}
    \begin{columns}[t]
    \column{.5\textwidth}
    \centering
    \includegraphics[width=4.1cm]{FourthJetPt2Tau2SS.png}\\
    \includegraphics[width=4.1cm]{FifthJetPt2Tau2SS.png}
    \column{.5\textwidth}
    \centering
    \includegraphics[width=4.1cm]{SixthJetPt2Tau2SS.png}\\
    \includegraphics[width=4.1cm]{Centrality2Tau2SS.png}
    \end{columns}
\end{frame}
\begin{frame}{2Tau2SS}
    \begin{columns}[t]
    \column{.5\textwidth}
    \centering
    \includegraphics[width=4.1cm]{HT2Tau2SS.png}\\
    \includegraphics[width=4.1cm]{MHTDividedByMET2Tau2SS.png}
    \column{.5\textwidth}
    \centering
    \includegraphics[width=4.1cm]{MaxDeltaRJets2Tau2SS.png}\\
    \includegraphics[width=4.1cm]{MinDeltaRJets2Tau2SS.png}
    \end{columns}
\end{frame}
\begin{frame}{2Tau2SS}
    \begin{columns}[t]
    \column{.5\textwidth}
    \centering
    \includegraphics[width=4.1cm]{SecondTopPt2Tau2SS.png}\\
    \includegraphics[width=4.1cm]{NumSelBJetsM2Tau2SS.png}
    \column{.5\textwidth}
    \centering
    \includegraphics[width=4.1cm]{NumofTops2Tau2SS.png}\\
    \includegraphics[width=4.1cm]{LeadingTopPt2Tau2SS.png}
%    \includegraphics[width=4.1cm]{Met_pt2Tau2SS.png}\\%Met_pt1Tau2OS.png
%    \includegraphics[width=4.1cm]{2Tau2SS.png}
    \end{columns}
\end{frame}


\begin{frame}{Significance}
    \begin{table}[htbp] %[h]
    \centering
%    \small
    \footnotesize%Use \footnotesize for a 20% (linear) reduction in font size
    \setlength\tabcolsep{2pt}%Reduce the amount of intercolumn whitespace
%    \begin{tabular}{|l | l | c|}
    \resizebox{\textwidth}{!}{% 
%        \begin{tabular}{|l | l |>{$}c<{$}|} 
        \begin{tabular}{|c | c |c|} 
         \hline
         process & event selection &  significance \\% [0.5ex] %\[1ex]This adds extra space to the cell
         \hline
         \hline
        1Tau0L & Number of Jets>7, Number of b jets>2, Number of tops>1,invariant mass of jets>800   &  \\
        1Tau1L &   &  \\
        1Tau2OS &   &  \\
        1Tau2SS &   &  \\
        1Tau3L &   &  \\
        2Tau0L &   &  \\
        2Tau1L &   &  \\
        2Tau2OS &   &  \\
        2Tau2SS &   &  \\
        
%         TT &
%         WJets&
%         DY &
%         ttbb &
%         WZ &
%         WW &
%         ZZ &
%         WG &
%         single top &
%         ggFH &
         \hline
        \end{tabular}
    }
    \caption{Subchannel}
    \label{table:4}
    \end{table}   
\end{frame}










\begin{frame}{Next Step}
    \begin{itemize}
    \item
       Optimize object selection
    \item
        Optimize event selection
    \item
        Use data driven to estimate TT, DY and W + jets.
    \item
       Learn machine learning methed to do event selection.
    \end{itemize}
\end{frame}

\begin{frame}{Significance}
    \begin{itemize}
    \item
       d
    \item
        d
    \end{itemize}
    \begin{table}[htbp] %[h]
    \centering
%    \small
    \footnotesize%Use \footnotesize for a 20% (linear) reduction in font size
    \setlength\tabcolsep{2pt}%Reduce the amount of intercolumn whitespace
%    \begin{tabular}{|l | l | c|}
    \resizebox{\textwidth}{!}{% 
%        \begin{tabular}{|l | l |>{$}c<{$}|} 
        \begin{tabular}{|c | c |c|} 
         \hline
         process & preselection &  step1 \\% [0.5ex] %\[1ex]This adds extra space to the cell
         \hline
         \hline
%         TT &
%         WJets&
%         DY &
%         ttbb &
%         WZ &
%         WW &
%         ZZ &
%         WG &
%         single top &
%         ggFH &
         2Tau 3L & 3 tight lepton & 2 medium tau\\
         \hline
        \end{tabular}
    }
    \caption{Subchannel}
    \label{table:4}
    \end{table}   
\end{frame}



\begin{frame}{Optimazation}
    \begin{itemize}
    \item
       d
    \item
        d
    \end{itemize}
\end{frame}


\subsection{Selection With Categorization}


\begin{frame}{}
    \begin{itemize}
    \item
       d
    \item
        d
    \end{itemize}
\end{frame}

\begin{frame}{Make Titles Informative.}
    \begin{itemize}
    \item
       d
    \item
        d
    \end{itemize}
\end{frame}

\begin{frame}{Make Titles Informative.}
    \begin{itemize}
    \item
       d
    \item
        d
    \end{itemize}
\end{frame}



\section*{Summary}

\begin{frame}{Summary}
  % Keep the summary *very short*.
  \begin{itemize}
  \item
    The \alert{first main message} of your talk in one or two lines.
  \item
    The \alert{second main message} of your talk in one or two lines.
  \item
    Perhaps a \alert{third message}, but not more than that.
  \end{itemize}
  % The following outlook is optional.
  \vskip0pt plus.5fill
  \begin{itemize}
  \item
    Outlook
    \begin{itemize}
    \item
      Something you haven't solved.
    \item
      Something else you haven't solved.
    \end{itemize}
  \end{itemize}
\end{frame}

\begin{frame}
\frametitle{Sample frame title}

In this slide, some important text will be
\alert{highlighted} because it's important.
Please, don't abuse it.

\begin{block}{Remark}
Sample text
\end{block}

\begin{alertblock}{Important theorem}
Sample text in red box
\end{alertblock}

\begin{examples}
Sample text in green box. The title of the block is ``Examples".
\end{examples}
\end{frame}

\begin{frame}{Make Titles Informative.}
  You can create overlays\dots
  \begin{itemize}
  \item using the \texttt{pause} command:
    \begin{itemize}
    \item
      First item.
      \pause
    \item    
      Second item.
    \end{itemize}
  \item
    using overlay specifications:
    \begin{itemize}
    \item<3->
      First item.
    \item<4->
      Second item.
    \end{itemize}
  \item
    using the general \texttt{uncover} command:
    \begin{itemize}
      \uncover<5->{\item
        First item.}
      \uncover<6->{\item
        Second item.}
    \end{itemize}
  \end{itemize}
\end{frame}


% All of the following is optional and typically not needed. 
\appendix
\section<presentation>*{\appendixname}
\subsection<presentation>*{For Further Reading}

\begin{frame}[allowframebreaks]
  \frametitle<presentation>{For Further Reading}
    
  \begin{thebibliography}{10}
    
  \beamertemplatebookbibitems
  % Start with overview books.

  \bibitem{Author1990}
    A.~Author.
    \newblock {\em Handbook of Everything}.
    \newblock Some Press, 1990.
 
    
  \beamertemplatearticlebibitems
  % Followed by interesting articles. Keep the list short. 

  \bibitem{Someone2000}
    S.~Someone.
    \newblock On this and that.
    \newblock {\em Journal of This and That}, 2(1):50--100,
    2000.
  \end{thebibliography}
\end{frame}

\end{document}


