\documentclass{beamer}

% This file is a solution template for:

% - Talk at a conference/colloquium.
% - Talk length is about 20min.
% - Style is ornate.



% Copyright 2004 by Till Tantau <tantau@users.sourceforge.net>.
%
% In principle, this file can be redistributed and/or modified under
% the terms of the GNU Public License, version 2.
%
% However, this file is supposed to be a template to be modified
% for your own needs. For this reason, if you use this file as a
% template and not specifically distribute it as part of a another
% package/program, I grant the extra permission to freely copy and
% modify this file as you see fit and even to delete this copyright
% notice. 


\mode<presentation> %
{
  %\usetheme{Warsaw}
  % or ...
%\usetheme{Madrid}%https://www.overleaf.com/learn/latex/Beamer#Reference_guide
  \usetheme{AnnArbor}

  \setbeamercovered{transparent}
  % or whatever (possibly just delete it)
}

\usepackage{array}
\usepackage{tabularx} 
\usepackage{underscore}
%\usepackage{multirow}
\usepackage{longtable}


\usepackage[english]{babel}
% or whatever

\usepackage[latin1]{inputenc}
% or whatever

\usepackage{times}
\usepackage[T1]{fontenc}
% Or whatever. Note that the encoding and the font should match. If T1
% does not look nice, try deleting the line with the fontenc.


\title[IHEP Group Meeting] % (optional, use only with long paper titles)
{Progress Report on Tau Final States of TTTT}
%\subtitle
%{Include Only If Paper Has a Subtitle}
\author[Huiling Hua] % (optional, for multiple authors)
%{A.~B.~Arthur\inst{1} \and J.~Doe\inst{2}}
{Huiling Hua\inst{1} \and Hongbo Liao\inst{1} \and Hideki Okawa\inst{2} \and Yu    Zhang\inst{2}}
%\author{Huiling Hua}
%\institute{IHEP}
\institute[IHEP] % (optional)
{
  \inst{1}%
 % Faculty of Physics\\
 % Very Famous University
    IHEP
  \and
  \inst{2}%
    Fudan University
}
% - Give the names in the same order as the appear in the paper.
% - Use the \inst{?} command only if the authors have different
% - Use the \inst command only if there are several affiliations.
% - Keep it simple, no one is interested in your street address.
\date[IHEP 2020] % (optional, should be abbreviation of conference name)
{IHEP Group Meeting, 2020}
% - Either use conference name or its abbreviation.
% - Not really informative to the audience, more for people (including
%   yourself) who are reading the slides online
\subject{Physics Analysis}
% This is only inserted into the PDF information catalog. Can be left
% out. 

% If you have a file called "university-logo-filename.xxx", where xxx
% is a graphic format that can be processed by latex or pdflatex,
% resp., then you can add a logo as follows:
% \pgfdeclareimage[height=0.5cm]{university-logo}{university-logo-filename}
% \logo{\pgfuseimage{university-logo}}

% Delete this, if you do not want the table of contents to pop up at
% the beginning of each subsection:
\AtBeginSubsection[]
{
  \begin{frame}<beamer>{Outline}
    \tableofcontents[currentsection,currentsubsection]
  \end{frame}
}

% If you wish to uncover everything in a step-wise fashion, uncomment
% the following command: 
%\beamerdefaultoverlayspecification{<+->}


\begin{document}

\begin{frame}
  \titlepage
\end{frame}

\begin{frame}{Outline}
  \tableofcontents
  % You might wish to add the option [pausesections]
\end{frame}


% Structuring a talk is a difficult task and the following structure
% may not be suitable. Here are some rules that apply for this
% solution: 

% - Exactly two or three sections (other than the summary).
% - At *most* three subsections per section.
% - Talk about 30s to 2min per frame. So there should be between about
%   15 and 30 frames, all told.

% - A conference audience is likely to know very little of what you
%   are going to talk about. So *simplify*!
% - In a 20min talk, getting the main ideas across is hard
%   enough. Leave out details, even if it means being less precise than
%   you think necessary.
% - If you omit details that are vital to the proof/implementation,
%   just say so once. Everybody will be happy with that.

\section{Motivation}
\subsection{Introduction to 4 Tops Process}

%\begin{frame}{Make Titles Informative. Use Uppercase Letters.}{Subtitles are optional.}
\begin{frame}{4 Tops Process}%{Sunbtitles are optional.}
  % - A title should summarize the slide in an understandable fashion
  %   for anyone how does not follow everything on the slide itself.
  \begin{itemize}
  \item
    1% Use \texttt{itemize} a lot.
  \item
    2%Use very short sentences or short phrases.
  \end{itemize}
\end{frame}


\subsection{Previous Work}
\begin{frame}{Make Titles Informative.}
\end{frame}


\section{CutBased Selection}


\begin{frame}{Data and MC samples}
    \begin{itemize}
    \item
    CMSSW version: CMSSW_10_2_20_UL
    \end{itemize}
    \begin{table}[htbp] %[h]
    \centering
%    \small
    \footnotesize%Use \footnotesize for a 20% (linear) reduction in font size
    \setlength\tabcolsep{2pt}%Reduce the amount of intercolumn whitespace
%    \begin{tabular}{|l | l | c|}
    \resizebox{\textwidth}{!}{% 
        \begin{tabular}{|l | l |>{$}c<{$}|} 
         \hline
         Process & Sample Name & Cross Section[pb]  \\% [0.5ex] %\[1ex]This adds extra space to the cell
         \hline
         \hline
         TTTT & TTTT_TuneCUETP8M2T4_13TeV-amcatnlo-pythia8 & 9.103\times10^{-3} \\ %9.103e-03 +- 1.401e-05 pb
         TTJets &  TTJets_TuneCUETP8M2T4_13TeV-amcatnloFXFX-pythia8 & 7.467\times10^{2} \\   %7.467e+02 +- 2.820e+00 pb
         TTGJets &  TTGJets_TuneCUETP8M1_13TeV-amcatnloFXFX-madspin-pythia8 &3.773\times10^{0}\\      %3.773e+00 +- 1.178e-02 pb
         ttZJets&  ttZJets_13TeV_madgraphMLM-pythia8& 6.559\times10{-1} \\      %6.559e-01 +- 5.438e-04 p
         ttWJets&  ttWJets_13TeV_madgraphMLM  & 2.014\times10^{-1} \\     %2.014e-01 +- 2.461e-03 pb
         ttH &  ttH_4f_ctcvcp_TuneCP5_13TeV_madgraph_pythia8 & 3.372\times10^{-1}  \\ %3.372e-01 +- 6.246e-05 pb
         ttbb &  ttbb_4FS_ckm_amcatnlo_madspin_pythia8 &1.393\times10^{1} \\     %1.393e+01 +- 3.629e-02 pb
        WZ &  WZ_TuneCUETP8M1_13TeV-pythia8& 2.343\times10^{1} \\      %2.343e+01 +- 1.049e-02 pb
         WW&  WW_TuneCUETP8M1_13TeV-pythia8 & 6.430\times10^{1}\\      %6.430e+01 +- 2.817e-02 pb
          WpWpJJ&  WpWpJJ_EWK-QCD_TuneCUETP8M1_13TeV-madgraph-pythia8 & 5.390\times10^{-2} \\      %5.390e-02 +- 2.905e-05 pb
         ZZ&  ZZ_TuneCUETP8M1_13TeV-pythia8 & 1.016\times10^{1} \\     %1.016e+01 +- 5.141e-03 pb
        WG &  WGJets_MonoPhoton_PtG-40to130_TuneCUETP8M1_13TeV-madgraph & 1.269\times10^{1}  \\      %1.269e+01 +- 1.038e-02 pb
         ZG&  ZGJetsToLLG_EW_LO_13TeV-sherpa & 1.319\times10^{-1} \\       %1.319e-01 +- 1.454e-04 pb
        WWW&  WWW_4F_TuneCUETP8M1_13TeV-amcatnlo-pythia8& 2.086\times10^{-1}\\        %2.086e-01 +- 2.024e-04 pb
         WWZ&  WWZ_TuneCUETP8M1_13TeV-amcatnlo-pythia8 & 1.651\times10^{-1}\\       %1.651e-01 +- 1.724e-04 pb
       WWG &  WWG_TuneCUETP8M1_13TeV-amcatnlo-pythia8 & 2.147\times10^{-1}\\      %2.147e-01 +- 2.206e-04 pb
       ZZZ &  ZZZ_TuneCUETP8M1_13TeV-amcatnlo-pythia8 & 1.398\times10^{-2}\\      %1.398e-02 +- 1.496e-05 pb
        WZZ&  WZZ_TuneCUETP8M1_13TeV-amcatnlo-pythia8 & 5.565\times10^{-2}\\      %5.565e-02 +- 5.500e-05 pb
%        WZG&  WZG_TuneCUETP8M1_13TeV-amcatnlo-pythia8 & 4.123\times10^{-2}\\      %4.123e-02 +- 4.251e-05 pb
%        WGG&  WGG_5f_TuneCUETP8M1_13TeV-amcatnlo-pythia8 & 1.819\times10^{0}\\      %1.819e+00 +- 5.227e-05 pb
%       WGG &  WGGJets_TuneCUETP8M1_13TeV_madgraphMLM_pythia8 & 1.711\times10^{0}\\      %1.711e+00 +- 1.753e-03 pb
%        ZGG&  ZGGJets_ZToHadOrNu_5f_LO_madgraph_pythia8 & 3.717\times10^{-1}\\      %3.717e-01 +- 4.788e-04 pb
%        WJets&  WJetsToLNu_TuneCUETP8M1_13TeV-madgraphMLM-pythia8       &5.030\times10^{+04} \\% 3.991e+01 pb
         \hline
        \end{tabular}
    }
    \caption{Table to test captions and labels}
    \label{table:1}
    \end{table}   
\end{frame}


\begin{frame}{Data and MC samples}
    \begin{table}[htbp] %[h]
    \centering
%    \small
    \footnotesize%Use \footnotesize for a 20% (linear) reduction in font size
    \setlength\tabcolsep{2pt}%Reduce the amount of intercolumn whitespace
%    \begin{tabular}{|l | l | c|}
    \resizebox{\textwidth}{!}{% 
        \begin{tabular}{|l | l |>{$}c<{$}|} 
         \hline
         Process & Sample Name & Cross Section[pb]  \\% [0.5ex] %\[1ex]This adds extra space to the cell
         \hline
         \hline
        WZG&  WZG_TuneCUETP8M1_13TeV-amcatnlo-pythia8 & 4.123\times10^{-2}\\      %4.123e-02 +- 4.251e-05 pb
        WGG&  WGG_5f_TuneCUETP8M1_13TeV-amcatnlo-pythia8 & 1.819\times10^{0}\\      %1.819e+00 +- 5.227e-05 pb
       WGG &  WGGJets_TuneCUETP8M1_13TeV_madgraphMLM_pythia8 & 1.711\times10^{0}\\      %1.711e+00 +- 1.753e-03 pb
        ZGG&  ZGGJets_ZToHadOrNu_5f_LO_madgraph_pythia8 & 3.717\times10^{-1}\\      %3.717e-01 +- 4.788e-04 pb
        WJets&  WJetsToLNu_TuneCUETP8M1_13TeV-madgraphMLM-pythia8       &5.030\times10^{+4} \\% 3.991e+01 pb
      DY  &  DYJetsToTauTau_ForcedMuEleDecay_M-50_TuneCUETP8M1_13TeV-amcatnloFXFX-pythia8_ext1      &1.983\times10^{+3} \\   %%1.983e+03 +- 4.355e+00 pb
%        \multirow{4}{2em}{single top} &  tZq_ll_4f_ckm_NLO_TuneCP5_PSweights_13TeV-amcatnlo-pythia8       &7.358\times10^{-2} \\   %%7.358e-02 +- 1.966e-04 pb
        \hline
        single top &  tZq_ll_4f_ckm_NLO_TuneCP5_PSweights_13TeV-amcatnlo-pythia8       &7.358\times10^{-2} \\   %%7.358e-02 +- 1.966e-04 pb
                   &  tZq_nunu_4f_13TeV-amcatnlo-pythia8_TuneCUETP8M1 &0000       \\   %
                   &  ST_tW_antitop_5f_inclusiveDecays_13TeV-powheg-pythia8_TuneCUETP8M2T4       &3.806\times10^{+1} \\   %%3.806e+01 +- 3.055e-02 pb
                   &  ST_tW_top_5f_inclusiveDecays_13TeV-powheg-pythia8_TuneCUETP8M2T4       &3.809\times10^{+1} \\   %%3.809e+01 +- 3.050e-02 pb
        \hline
         TG   &  TGJets_TuneCUETP8M1_13TeV_amcatnlo_madspin_pythia8       &2.967\times10^{+0} \\   %%2.967e+00 +- 1.052e-02 pb
        \hline
        TH &  THW_ctcvcp_HIncl_M125_TuneCP5_13TeV-madgraph-pythia8   &1.467\times10^{-1} \\   %%1.467e-01 +- 1.485e-05 pb
           &  THQ_ctcvcp_Hincl_13TeV-madgraph-pythia8_TuneCUETP8M1   &8.816\times10^{-1} \\   %%8.816e-01 +- 1.991e-04 pb
        Z/W+H &  VHToNonbb_M125_13TeV_amcatnloFXFX_madspin_pythia8   &2.137\times10^{+0} \\   %%2.137e+00 +- 5.124e-03 pb
              &  ZHToTauTau_M125_13TeV_powheg_pythia8   &7.524\times10^{-1} \\   %%7.524e-01 +- 3.643e-03 pb
              &  ZH_HToBB_ZToLL_M125_13TeV_powheg_pythia8   &7.523\times10^{-2} \\   %%7.523e-02 +- 3.138e-04 pb
        ggFH  &  GluGluHToZZTo4L_M125_13TeV_powheg2_JHUgenV6_pythia8   &2.999\times10^{+1} \\   %%2.999e+01 +- 2.112e-02 pb
              &  GluGluHToBB_M125_13TeV_amcatnloFXFX_pythia8   &3.210\times10^{+1} \\   %%3.210e+01 +- 9.353e-02 pb
              &  GluGluHToGG_M125_13TeV_amcatnloFXFX_pythia8   &3.198\times10^{+1} \\   %%3.198e+01 +- 9.594e-02 pb
              &  GluGluHToMuMu_M-125_TuneCP5_PSweights_13TeV_powheg_pythia8   &2.999\times10^{+1} \\   %%2.999e+01 +- 2.112e-02 pb
              &  GluGluHToTauTau_M125_13TeV_powheg_pythia8   &3.052\times10^{+1} \\   %%3.052e+01 +- 2.150e-02 pb
              &  GluGluHToWWTo2L2Nu_M125_13TeV_powheg_JHUgen_pythia8   &3.052\times10^{+1} \\   %%3.052e+01 +- 2.150e-02 pb
              &  GluGluHToWWToLNuQQ_M125_13TeV_powheg_JHUGenV628_pythia8   &2.999\times10^{+1} \\   %%2.999e+01 +- 2.112e-02 pb
%        VBFH  &  VBFHToWWToLNuQQ_M125_13TeV_powheg_JHUGenV628_pythia8   &3.769\times10^{+0} \\   %%3.769e+00 +- 1.589e-02 pb
%              &  VBFHToWWTo2L2Nu_M125_13TeV_powheg_JHUgenv628_pythia8   &3.769\times10^{+0} \\   %%3.769e+00 +- 1.589e-02 pb
%              &  VBFHToTauTau_M125_13TeV_powheg_pythia8   \\   %???3.721e+00 +/- 1.598e-02  final cross section = 0.000e+00 +-%???3.721e+00 +/- 1.598e-02  final cross section = 0.000e+00 +- 0.000e+00 pb
%          %???how can it the fianal cross section be 0?
%              &  VBFHToMuMu_M-125_TuneCP5_PSweights_13TeV_powheg_pythia8   \\   %???
%              &  VBFHToGG_M125_13TeV_amcatnlo_pythia8_v2   &3.992\times10^{+0} \\   %%3.992e+00 +- 8.933e-03 pb
%              &  VBFHToBB_M-125_13TeV_powheg_pythia8_weightfix   \\   %???
%              &  VBF_HToZZTo4L_M125_13TeV_powheg2_JHUgenV6_pythia8   &3.769\times10^{+0} \\   %%3.769e+00 +- 1.589e-02 pb
         \hline
        \end{tabular}
    }
    \caption{Table to test captions and labels}
    \label{table:2}
    \end{table}   
\end{frame}


\begin{frame}{Data and MC samples}
    \begin{table}[htbp] %[h]
    \centering
%    \small
    \footnotesize%Use \footnotesize for a 20% (linear) reduction in font size
    \setlength\tabcolsep{2pt}%Reduce the amount of intercolumn whitespace
%    \begin{tabular}{|l | l | c|}
    \resizebox{\textwidth}{!}{% 
        \begin{tabular}{|l | l |>{$}c<{$}|} 
         \hline
         Process & Sample Name & Cross Section[pb]  \\% [0.5ex] %\[1ex]This adds extra space to the cell
         \hline
         \hline
        VBFH  &  VBFHToWWToLNuQQ_M125_13TeV_powheg_JHUGenV628_pythia8   &3.769\times10^{+0} \\   %%3.769e+00 +- 1.589e-02 pb
              &  VBFHToWWTo2L2Nu_M125_13TeV_powheg_JHUgenv628_pythia8   &3.769\times10^{+0} \\   %%3.769e+00 +- 1.589e-02 pb
              &  VBFHToTauTau_M125_13TeV_powheg_pythia8   \\   %???3.721e+00 +/- 1.598e-02  final cross section = 0.000e+00 +-%???3.721e+00 +/- 1.598e-02  final cross section = 0.000e+00 +- 0.000e+00 pb
          %???how can it the fianal cross section be 0?
              &  VBFHToMuMu_M-125_TuneCP5_PSweights_13TeV_powheg_pythia8   \\   %???
              &  VBFHToGG_M125_13TeV_amcatnlo_pythia8_v2   &3.992\times10^{+0} \\   %%3.992e+00 +- 8.933e-03 pb
              &  VBFHToBB_M-125_13TeV_powheg_pythia8_weightfix   \\   %???
              &  VBF_HToZZTo4L_M125_13TeV_powheg2_JHUgenV6_pythia8   &3.769\times10^{+0} \\   %%3.769e+00 +- 1.589e-02 pb
         \hline
        \end{tabular}
    }
    \caption{Table to test captions and labels}
    \label{table:2}
    \end{table}   
\end{frame}



%\begin{frame}{Data and MC samples}
%    \begin{itemize}
%    \item
%    CMSSW version: CMSSW_10_2_20_UL
%    \end{itemize}
%    \begin{table}[htbp] %[h]
%    \centering
%%    \small
%    \footnotesize%Use \footnotesize for a 20% (linear) reduction in font size
%    \setlength\tabcolsep{2pt}%Reduce the amount of intercolumn whitespace
%%    \begin{tabular}{|l | l | c|}
%    \resizebox{\textwidth}{!}{% 
%%        \begin{tabular}{|l | l |>{$}c<{$}|} 
%        \begin{longtable}{|l | l |>{$}c<{$}|} 
%         \hline
%         Process & Sample Name & Cross Section[pb]  \\% [0.5ex] %\[1ex]This adds extra space to the cell
%         \hline
%        \endfirsthead
%
%
%%         \hline
%         TTTT & TTTT_TuneCUETP8M2T4_13TeV-amcatnlo-pythia8 & 9.103\times10^{-3} \\ %9.103e-03 +- 1.401e-05 pb
%         TTJets &  TTJets_TuneCUETP8M2T4_13TeV-amcatnloFXFX-pythia8 & 746.7\\   %7.467e+02 +- 2.820e+00 pb
%         TTGJets &  TTGJets_TuneCUETP8M1_13TeV-amcatnloFXFX-madspin-pythia8 &3.77\\      %3.773e+00 +- 1.178e-02 pb
%         ttZJets&  ttZJets_13TeV_madgraphMLM-pythia8& 0.6559 \\      %6.559e-01 +- 5.438e-04 p
%         ttWJets&  ttWJets_13TeV_madgraphMLM  & 0.2014\\     %2.014e-01 +- 2.461e-03 pb
%         ttH &  ttH_4f_ctcvcp_TuneCP5_13TeV_madgraph_pythia8 & 0.3372  \\ %3.372e-01 +- 6.246e-05 pb
%         ttbb &  ttbb_4FS_ckm_amcatnlo_madspin_pythia8 &13.93 \\     %1.393e+01 +- 3.629e-02 pb
%        WZ &  WZ_TuneCUETP8M1_13TeV-pythia8& 23.43\\      %2.343e+01 +- 1.049e-02 pb
%         WW&  WW_TuneCUETP8M1_13TeV-pythia8 &64.30\\      %6.430e+01 +- 2.817e-02 pb
%          WpWpJJ&  WpWpJJ_EWK-QCD_TuneCUETP8M1_13TeV-madgraph-pythia8 & 0.05390\\      %5.390e-02 +- 2.905e-05 pb
%         ZZ&  ZZ_TuneCUETP8M1_13TeV-pythia8 & 10.16\\     %1.016e+01 +- 5.141e-03 pb
%        WG &  WGJets_MonoPhoton_PtG-40to130_TuneCUETP8M1_13TeV-madgraph & 12.69\\      %1.269e+01 +- 1.038e-02 pb
%         ZG&  ZGJetsToLLG_EW_LO_13TeV-sherpa & 1.319 \\       %1.319e-01 +- 1.454e-04 pb
%        WWW&  WWW_4F_TuneCUETP8M1_13TeV-amcatnlo-pythia8& 2.086\times10^{-1}\\        %2.086e-01 +- 2.024e-04 pb
%         WWZ&  WWZ_TuneCUETP8M1_13TeV-amcatnlo-pythia8 & 1.651\times10^{-1}\\       %1.651e-01 +- 1.724e-04 pb
%       WWG &  WWG_TuneCUETP8M1_13TeV-amcatnlo-pythia8 & 2.147\times10^{-1}\\      %2.147e-01 +- 2.206e-04 pb
%       ZZZ &  ZZZ_TuneCUETP8M1_13TeV-amcatnlo-pythia8 & 1.398\times10^{-2}\\      %1.398e-02 +- 1.496e-05 pb
%        WZZ&  WZZ_TuneCUETP8M1_13TeV-amcatnlo-pythia8 & 5.565\times10^{-2}\\      %5.565e-02 +- 5.500e-05 pb
%        WZG&  WZG_TuneCUETP8M1_13TeV-amcatnlo-pythia8 & 4.123\times10^{-2}\\      %4.123e-02 +- 4.251e-05 pb
%        WGG&  WGG_5f_TuneCUETP8M1_13TeV-amcatnlo-pythia8 & 1.819\times10^{0}\\      %1.819e+00 +- 5.227e-05 pb
%       WGG &  WGGJets_TuneCUETP8M1_13TeV_madgraphMLM_pythia8 & 1.711\times10^{0}\\      %1.711e+00 +- 1.753e-03 pb
%        ZGG&  ZGGJets_ZToHadOrNu_5f_LO_madgraph_pythia8 & 3.717\times10^{-1}\\      %3.717e-01 +- 4.788e-04 pb
%%        WJets&  WJetsToLNu_TuneCUETP8M1_13TeV-madgraphMLM-pythia8       &5.030\times10^{+04} \\% 3.991e+01 pb
%         \hline
%    \end{longtable}
%    }
%    \caption{Table to test captions and labels}
%    \label{table:1}
%    \end{table}   
%\end{frame}




\subsection{Selection Without Categorization}
\begin{frame}{Object Selection}
  \begin{itemize}
  \item
    Electron 
    \begin{itemize}
    \item
        pt>20, |eta|<2.4
    \item
        electron cut based loose ID(Fall-94X-V2)
    \end{itemize}
  \item
    Muon
    \begin{itemize}
    \item
        pt>20, |eta|<2.4
    \item
        loose ID (cutbased, recommended by muon POG)
    \item
        pass loose isolation(same as SS of TTTT)
    \end{itemize}
  \item
    Tau
    \begin{itemize}
    \item
        pt>20, |eta|<2.3
    \item
        loose tau ID(same as ttH)
    \end{itemize}
  \end{itemize}
\end{frame}


\begin{frame}{Object Selection}
  \begin{itemize}
  \item
    Jet
    \begin{itemize}
    \item
        pt>25
    \item
        loose jet(recommended by JETMET)
    \end{itemize}
  \item
     B Jet
     \begin{itemize}
        \item
            use Deep Flavour B tagging algorithm 
        \item 
            use the recommended working points
     \end{itemize}
  \item
      Top 
    \begin{itemize}
    \item
        use SUSY HOT TopTagger 
    \item 
        resolved
    \end{itemize}

  \end{itemize}
\end{frame}


\begin{frame}{Event Selection-Preselection}
    \begin{itemize}
    \item
        MET filters
    \item
        at least 1 loose tau
    \end{itemize}
\end{frame}


\begin{frame}{Distribution of Signal Vs Background}
    \begin{itemize}
    \item
      1 
    \item
       d 
    \end{itemize}
\end{frame}



\begin{frame}{Significance}
    \begin{itemize}
    \item
       d
    \item
        d
    \end{itemize}
\end{frame}



\begin{frame}{Optimazation}
    \begin{itemize}
    \item
       d
    \item
        d
    \end{itemize}
\end{frame}


\subsection{Selection With Categorization}


\begin{frame}{}
    \begin{itemize}
    \item
       d
    \item
        d
    \end{itemize}
\end{frame}

\begin{frame}{Make Titles Informative.}
    \begin{itemize}
    \item
       d
    \item
        d
    \end{itemize}
\end{frame}

\begin{frame}{Make Titles Informative.}
    \begin{itemize}
    \item
       d
    \item
        d
    \end{itemize}
\end{frame}



\section*{Summary}

\begin{frame}{Summary}
  % Keep the summary *very short*.
  \begin{itemize}
  \item
    The \alert{first main message} of your talk in one or two lines.
  \item
    The \alert{second main message} of your talk in one or two lines.
  \item
    Perhaps a \alert{third message}, but not more than that.
  \end{itemize}
  % The following outlook is optional.
  \vskip0pt plus.5fill
  \begin{itemize}
  \item
    Outlook
    \begin{itemize}
    \item
      Something you haven't solved.
    \item
      Something else you haven't solved.
    \end{itemize}
  \end{itemize}
\end{frame}

\begin{frame}
\frametitle{Sample frame title}

In this slide, some important text will be
\alert{highlighted} because it's important.
Please, don't abuse it.

\begin{block}{Remark}
Sample text
\end{block}

\begin{alertblock}{Important theorem}
Sample text in red box
\end{alertblock}

\begin{examples}
Sample text in green box. The title of the block is ``Examples".
\end{examples}
\end{frame}

\begin{frame}{Make Titles Informative.}
  You can create overlays\dots
  \begin{itemize}
  \item using the \texttt{pause} command:
    \begin{itemize}
    \item
      First item.
      \pause
    \item    
      Second item.
    \end{itemize}
  \item
    using overlay specifications:
    \begin{itemize}
    \item<3->
      First item.
    \item<4->
      Second item.
    \end{itemize}
  \item
    using the general \texttt{uncover} command:
    \begin{itemize}
      \uncover<5->{\item
        First item.}
      \uncover<6->{\item
        Second item.}
    \end{itemize}
  \end{itemize}
\end{frame}


% All of the following is optional and typically not needed. 
\appendix
\section<presentation>*{\appendixname}
\subsection<presentation>*{For Further Reading}

\begin{frame}[allowframebreaks]
  \frametitle<presentation>{For Further Reading}
    
  \begin{thebibliography}{10}
    
  \beamertemplatebookbibitems
  % Start with overview books.

  \bibitem{Author1990}
    A.~Author.
    \newblock {\em Handbook of Everything}.
    \newblock Some Press, 1990.
 
    
  \beamertemplatearticlebibitems
  % Followed by interesting articles. Keep the list short. 

  \bibitem{Someone2000}
    S.~Someone.
    \newblock On this and that.
    \newblock {\em Journal of This and That}, 2(1):50--100,
    2000.
  \end{thebibliography}
\end{frame}

\end{document}


